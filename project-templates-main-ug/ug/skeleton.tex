% UG project example file, February 2022
%   A minior change in citation, September 2023 [HS]
% Do not change the first two lines of code, except you may delete "logo," if causing problems.
% Understand any problems and seek approval before assuming it's ok to remove ugcheck.
\documentclass[logo,bsc,singlespacing,parskip]{infthesis}
\usepackage{ugcheck}

% Include any packages you need below, but don't include any that change the page
% layout or style of the dissertation. By including the ugcheck package above,
% you should catch most accidental changes of page layout though.

\usepackage{microtype} % recommended, but you can remove if it causes problems
\usepackage{biblatex} 
\usepackage[british]{babel}
\usepackage{csquotes}
\DeclareFieldFormat{urldate}{%
  (Accessed: \thefield{urlday}/\thefield{urlmonth}/\thefield{urlyear})}
\usepackage{url}
\usepackage{amsmath, amsfonts, amssymb, amsthm}
\usepackage{mathrsfs}
\usepackage{tikz-cd}
\usepackage{hyperref}
% TODO: better colours
\usepackage[nameinlink]{cleveref}
\usepackage{enumitem}
\setlist{nosep}

\addbibresource{mybibfile.bib}

\theoremstyle{definition}
\newtheorem{prop}{Proposition}
\newtheorem{defn}[prop]{Definition}


\newcommand{\cat}[1]{\mathscr{#1}}
\newcommand{\ob}[1]{\obj(\mathscr{#1})}
\DeclareMathOperator{\obj}{ob}

\begin{document}
\begin{preliminary}

\title{Parsing Commutative Diagrams}

\author{Christopher Wilson}

% CHOOSE YOUR DEGREE a):
% please leave just one of the following un-commented
%\course{Artificial Intelligence}
%\course{Artificial Intelligence and Computer Science}
%\course{Artificial Intelligence and Mathematics}
%\course{Artificial Intelligence and Software Engineering}
%\course{Cognitive Science}
%\course{Computer Science}
%\course{Computer Science and Management Science}
\course{Computer Science and Mathematics}
%\course{Computer Science and Physics}
%\course{Software Engineering}
%\course{Master of Informatics} % MInf students

% CHOOSE YOUR DEGREE b):
% please leave just one of the following un-commented
%\project{MInf Project (Part 1) Report}  % 4th year MInf students
%\project{MInf Project (Part 2) Report}  % 5th year MInf students
\project{4th Year Project Report}        % all other UG4 students


\date{\today}

\abstract{
This skeleton demonstrates how to use the \texttt{infthesis} style for
undergraduate dissertations in the School of Informatics. It also emphasises the
page limit, and that you must not deviate from the required style.
The file \texttt{skeleton.tex} generates this document and should be used as a
starting point for your thesis. Replace this abstract text with a concise
summary of your report.
}

\maketitle

\newenvironment{ethics}
   {\begin{frontenv}{Research Ethics Approval}{\LARGE}}
   {\end{frontenv}\newpage}

\begin{ethics}
\textbf{Instructions:} \emph{Agree with your supervisor which
statement you need to include. Then delete the statement that you are not using,
and the instructions in italics.\\
\textbf{Either complete and include this statement:}}\\ % DELETE THESE INSTRUCTIONS
%
% IF ETHICS APPROVAL WAS REQUIRED:
This project obtained approval from the Informatics Research Ethics committee.\\
Ethics application number: ???\\
Date when approval was obtained: YYYY-MM-DD\\
%
\emph{[If the project required human participants, edit as appropriate, otherwise delete:]}\\ % DELETE THIS LINE
The participants' information sheet and a consent form are included in the appendix.\\
%
% IF ETHICS APPROVAL WAS NOT REQUIRED:
\textbf{\emph{Or include this statement:}}\\ % DELETE THIS LINE
This project was planned in accordance with the Informatics Research
Ethics policy. It did not involve any aspects that required approval
from the Informatics Research Ethics committee.

\standarddeclaration
\end{ethics}


\begin{acknowledgements}
Any acknowledgements go here.
\end{acknowledgements}


\tableofcontents
\end{preliminary}


\chapter{Introduction}

The preliminary material of your report should contain:
\begin{itemize}
\item
The title page.
\item
An abstract page.
\item
Declaration of ethics and own work.
\item
Optionally an acknowledgements page.
\item
The table of contents.
\end{itemize}

As in this example \texttt{skeleton.tex}, the above material should be
included between:
\begin{verbatim}
\begin{preliminary}
    ...
\end{preliminary}
\end{verbatim}
This style file uses roman numeral page numbers for the preliminary material.

The main content of the dissertation, starting with the first chapter,
starts with page~1. \emph{\textbf{The main content must not go beyond page~40.}}

The report then contains a bibliography and any appendices, which may go beyond
page~40. The appendices are only for any supporting material that's important to
go on record. However, you cannot assume markers of dissertations will read them.

You may not change the dissertation format (e.g., reduce the font size, change
the margins, or reduce the line spacing from the default single spacing). Be
careful if you copy-paste packages into your document preamble from elsewhere.
Some \LaTeX{} packages, such as \texttt{fullpage} or \texttt{savetrees}, change
the margins of your document. Do not include them!

Over-length or incorrectly-formatted dissertations will not be accepted and you
would have to modify your dissertation and resubmit. You cannot assume we will
check your submission before the final deadline and if it requires resubmission
after the deadline to conform to the page and style requirements you will be
subject to the usual late penalties based on your final submission time.

\section{Using Sections}

Divide your chapters into sub-parts as appropriate.

\section{Citations}

Citations, such as \autocite{P1} or \autocite{P2}, can be generated using
\texttt{BibTeX}. We recommend using the \texttt{natbib} package (default) or the newer \texttt{biblatex} system. 

You may use any consistent reference style that you prefer, including ``(Author, Year)'' citations. 

\chapter{Background}
In this chapter, we give a brief introduction to the mathematical field of Category Theory (Section \ref{bkg:cat}) and define a commutative diagram (Section \ref{bkg:diag}). We then discuss existing methods of drawing commutative diagrams (Section \ref{bkg:other}) and relevant graph drawing algorithms that can be used to draw commutative diagrams (Section \ref{bkg:graph-draw}).

\section{Category Theory}\label{bkg:cat}
Commutative diagrams are primarily used by mathematicians studying category theory, so to understand commutative diagrams and their uses it is helpful to know a bit about category theory.

When studying mathematics we encounter many different mathematical objects, such as groups, graphs, vector spaces, sets, natural numbers, and rings, all with their own rules and structure. There are quite often similarities and relationships between these objects, however, when working directly with these objects these similarities and relationships can sometimes be hard to see through the different notation's and ways of describing them, especially if the objects come from different areas of mathematics. Category theory aims to provide a framework for expressing every mathematical object and its structure, abstracting away the details so the similarities and relationships become more clear. This is done by defining a new mathematical object: the category.

\begin{defn}[{\autocite[Def 1.1.1]{leinster2016basic}}]\label{def:cat}
    A category $\cat{A}$ consists of:
    \begin{itemize}
        \item a collection $\ob{A}$ of objects;
        \item for each $A, B \in \ob{A}$, a collection $\cat A(A,B)$ of maps or arrows or morphisms from $A$ to $B$;
        \item for each $A, B, C \in \ob{A}$, a function 
        \begin{align*}
            \cat{A}(B,C) \times \cat{A}(A,B) &\to \cat{A}(A,C) \\
            (g,f) &\mapsto g \circ f,
        \end{align*}
        called composition;
        \item for each $A \in \ob{A}$, an element $1_A$ of $\cat{A}(A,A)$, called the identity on $A$.
    \end{itemize}
    satisfying the following axioms:
    \begin{itemize}
        \item associativity: for each $f \in \cat{A}(A,B)$, $g \in \cat{A}(B,C)$, and $h \in \cat{A}(C,D)$, we have $(h \circ g) \circ f = h \circ (g \circ f)$;
        \item identity laws: for each $f \in \cat{A}(A,B)$, we have $f \circ 1_A = f = 1_B \circ f$.
    \end{itemize}
\end{defn}

An example of a category is $\mathbf{Set}$, where the objects are sets, and the maps are set functions between the sets. For another example, we could pick a set $A$ and represent it with a category $\cat{A}$ by simply letting $\ob{A} = A$. Since objects in a generic set have no intrinsic relationship with each other the maps in $\cat{A}$ will just be the required identity maps, so to make this example more interesting we can apply a preordering to a set $A$. We do this by for each pair of elements $a,b \in A$ deciding if $a \ge b$\footnote{The difference between a preorder and an order is that in an order if $a \ge b$ and $b \ge a$ then $a = b$. This is not the case in a preorder.}. Let $\mathbf{a}, \mathbf{b} \in \ob{A}$ represent $a,b \in A$ respectively. We can represent this preorder in $\cat A$ by adding a map from $\mathbf{a}$ to $\mathbf{b}$ if $a \ge b$. The identity maps give us that $a \ge a$, and composition gives us the property that if $a \ge b$ and $b \ge c$, then $a \ge c$, as we can compose the map from $\mathbf{a}$ to $\mathbf{b}$ and the map from $\mathbf{b}$ to $\mathbf{c}$ to get a map from $\mathbf{a}$ to $\mathbf{c}$. 

Categories don't have to represent pure mathematical objects such as sets or numbers, for example, the category $\mathbf{Hask}$ \cite{wiki:hask} represents programs written in the language Haskell, with the objects being types and the maps being functions between those types. 

\section{Commutative Diagrams}\label{bkg:diag}
Consider a category $\cat B$, with objects $\ob{B} = \{A,B,C,D\}$ and non-identity maps $\cat{B}(A,B) = \{f\}$, $\cat{B}(A,C) = \{g\}$, $\cat{B}(B,C) = \{h\}$, $\cat{B}(D,B) = \{i\}$, $\cat{B}(D,C) = \{j\}$. Listing out all the non-identity maps like this does convey all the information we need, but in a way that's un-intuitive and hard to parse for a human reader, so let us find another way. Recall from the definition of a category (Definition \ref{def:cat}) that an alternative name for ``maps" are arrows. With this in mind, we draw each object spaced out on a blank page, and then for each pair of objects $(A', B')$ draw an arrow from $A'$ to $B'$ for each non-identity map from $A'$ to $B'$. We skip the identity maps because we know every object has one by definition, and in most circumstances we don't gain anything by seeing them, similar to how we often won't write $a + 0$, $a \cdot 1$, or $a/1$ but instead just $a$. Doing this to $\cat B$ gives us the diagram in Figure \ref{fig:comm-diag-ex1}.

\begin{figure}[h]
    \centering
    \begin{tikzcd}[cramped]
	A & B & D \\
	& C
	\arrow["f", from=1-1, to=1-2]
	\arrow["g"', from=1-1, to=2-2]
	\arrow["h", from=1-2, to=2-2]
	\arrow["i"', from=1-3, to=1-2]
	\arrow["j", from=1-3, to=2-2]
    \end{tikzcd}
    \caption{A commutative diagram representing category $\cat B$.}
    \label{fig:comm-diag-ex1}
\end{figure}

Informally, define a path from an object $A'$ to an object $B'$ to be a chain of consecutive arrows from $A'$ to $B'$. We say a diagram commutes, or is a \emph{commutative diagram} if and only if for every pair of objects every path between those two objects is equivalent to every other path. For example, the diagram in Figure \ref{fig:comm-diag-ex1} is commutative if $h \circ f = g$ and $h \circ i = j$, or in the other direction if we were shown Figure \ref{fig:comm-diag-ex1} and told it was commutative, then we would know that $h \circ f = g$ and $h \circ i = j$.

%Add stuff about representing theorems (e.g. correspondence theorem) with comm diags?

%This paragraph could be moved to "Existing Commutative Diagmrams..." to point out how none of the exsiting methods solve this problem.
When drawing a commutative diagram it can be hard to know where to initially position the objects, and a bad positioning can cause the diagram to be hard to read - defeating its purpose. This project aims to automate the creation of commutative diagrams from a list of maps, alleviating this problem.

\section{Existing Commutative Diagram Drawing Methods}\label{bkg:other}
There are many existing methods of drawing commutative diagrams, from hand-drawn to dedicated software. This section discusses some of these methods and the benefits and drawbacks of each.

\subsection{Hand-drawn}
Drawing commutative diagrams by hand is fairly easy and quick, but drawing a good looking diagram can be much harder and time-intensive. Arrows, for instance, take effort to draw uniformly, and curved arrows can be hard to draw regularly. It can also be hard to know where to initially position the objects to avoid creating a tangled web of intersecting arrows, especially in large and interconnected categories. 

\subsection{\LaTeX\ packages}
\LaTeX{} is a typesetting system frequently used by mathematicians to typeset mathematical documents. \LaTeX{} is recommended by the American Mathematical Society for authors who want to publish with them \cite{AMSlatexrec}, and a study in 2009 found that 96.9\% of submissions to 4 randomly selected mathematical journals were typeset in \LaTeX{} \cite{brischoux2009don}. Therefore, to look at how to create professional looking commutative diagrams, we should look at what \LaTeX{} has to offer.

The Comprehensive \TeX{} Archive, or CTAN, is the main repository of \LaTeX{} packages. Browsing the ``Commutative Diagrams'' topic \cite{ctancommdiag} gives us 16 packets, some of which build on other packets. All the packets ask us to position the objects ourselves, either using cartesian coordinates or by using a grid-structure. %include example of this?
In general, we have to define the starting position and direction of the arrows ourselves, but some packages such as CoDi \cite{Brasolin_2023} and DCpic \cite{Quaresma_2013} just ask us to define which objects the arrows go between.

The benefits and drawbacks of drawing commutative diagrams using these packages is almost inverse of hand-drawing, they produce professional looking diagrams, but the process of typing out the diagrams is much slower and more un-intuitive compared to the speed and simplicity of hand-drawing. This method also has the problem of a bad initial placement of objects leading to a messy diagram. The ease of fixing a bad placement relative to hand-drawing depends on the packet being used, ones that require us to define the position and direction of arrows will need large complicated re-writes, potentially from scratch, whereas packages that just require object positions will be fairly easy to re-organise. In conclusion, \LaTeX{} packages are good at producing professional looking commutative diagrams, but take much more effort to produce than hand-drawing.

\subsection{Graphical Editors}
Dedicated graphical editors for creating commutative diagrams with the tikzcd package \cite{Stoffel_2021}, such as quiver \cite{Arkor_quiver_2023} and tikzcd-editor, \cite{shen_tikzcd-editor_2023} mitigate the drawbacks of the \LaTeX{} packages by making the creation of the diagrams much easier. The graphical interface is much more intuitive to use than typing out the diagram manually, and the ability to drag objects helps mitigate the initial object problem by making modifications to the diagram trivially easy. 

\section{Graph Drawing Algorithms}\label{bkg:graph-draw}
There is very little existing research explicitly on algorithms for drawing commutative diagrams, but there is a lot of research on the topic of graph drawing. This section shows that we can use directed-graph drawing algorithms to draw commutative diagrams, and discusses the algorithm(s) that we (will) use to draw our commutative diagrams.

First, we define a directed graph.
\begin{defn}
    A directed graph $G$ contains two collections:
    \begin{enumerate}
        \item A non-empty finite set $V(G)$ of vertices, called the vertex set of $G$.
        \item A finite family $E(G)$ of ordered pairs of vertices $(v,w)$, where $v, w \in V(G)$, called edges. We say $(v,w)$ connects vertex $v$ to vertex $w$, but \emph{does not} connect vertex $w$ to vertex $v$.
    \end{enumerate}
\end{defn}
 A family is a set where we do not ignore duplicates, e.g. for families $\{1, 1\} \ne \{1\}$.

We can reduce the problem of commutative diagram drawing to directed graph drawing by treating each object to be drawn as a vertex and each map as an edge.
Unfortunately, drawing a graph nicely is NP-complete \cite{10.1007/3-540-52698-6_1}. A variety of algorithms for drawing directed graphs exist \cite{di1994algorithms}, and we use... Any background on whatever algorithm(s) I end up using will go here.


\section{Text Parsing}\label{bkg:text-parse}
To input the category being drawn, the user will enter the objects and morphisms in text form, using a languages we make for this purpose. We then need to parse this text. We do this by matching each chunk of text to a token. These tokens can then be converted into a graph, which apply our drawing algorithm to, then we can output a commutative diagram.

\chapter{Conclusions}

\section{Final Reminder}

The body of your dissertation, before the references and any appendices,
\emph{must} finish by page~40. The introduction, after preliminary material,
should have started on page~1.

You may not change the dissertation format (e.g., reduce the font size, change
the margins, or reduce the line spacing from the default single spacing). Be
careful if you copy-paste packages into your document preamble from elsewhere.
Some \LaTeX{} packages, such as \texttt{fullpage} or \texttt{savetrees}, change
the margins of your document. Do not include them!

Over-length or incorrectly-formatted dissertations will not be accepted and you
would have to modify your dissertation and resubmit. You cannot assume we will
check your submission before the final deadline and if it requires resubmission
after the deadline to conform to the page and style requirements you will be
subject to the usual late penalties based on your final submission time.

% \bibliographystyle{plain}
%\bibliographystyle{plainnat}
%\bibliography{mybibfile}
\printbibliography


% You may delete everything from \appendix up to \end{document} if you don't need it.
\appendix

\chapter{First appendix}

\section{First section}

Any appendices, including any required ethics information, should be included
after the references.

Markers do not have to consider appendices. Make sure that your contributions
are made clear in the main body of the dissertation (within the page limit).

\chapter{Participants' information sheet}

If you had human participants, include key information that they were given in
an appendix, and point to it from the ethics declaration.

\chapter{Participants' consent form}

If you had human participants, include information about how consent was
gathered in an appendix, and point to it from the ethics declaration.
This information is often a copy of a consent form.


\end{document}
